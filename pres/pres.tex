
% Default to the notebook output style

    


% Inherit from the specified cell style.




    
\documentclass{article}

    
    
    \usepackage{graphicx} % Used to insert images
    \usepackage{adjustbox} % Used to constrain images to a maximum size 
    \usepackage{color} % Allow colors to be defined
    \usepackage{enumerate} % Needed for markdown enumerations to work
    \usepackage{geometry} % Used to adjust the document margins
    \usepackage{amsmath} % Equations
    \usepackage{amssymb} % Equations
    \usepackage[mathletters]{ucs} % Extended unicode (utf-8) support
    \usepackage[utf8x]{inputenc} % Allow utf-8 characters in the tex document
    \usepackage{fancyvrb} % verbatim replacement that allows latex
    \usepackage{grffile} % extends the file name processing of package graphics 
                         % to support a larger range 
    % The hyperref package gives us a pdf with properly built
    % internal navigation ('pdf bookmarks' for the table of contents,
    % internal cross-reference links, web links for URLs, etc.)
    \usepackage{hyperref}
    \usepackage{longtable} % longtable support required by pandoc >1.10
    \usepackage{booktabs}  % table support for pandoc > 1.12.2
    

    
    
    \definecolor{orange}{cmyk}{0,0.4,0.8,0.2}
    \definecolor{darkorange}{rgb}{.71,0.21,0.01}
    \definecolor{darkgreen}{rgb}{.12,.54,.11}
    \definecolor{myteal}{rgb}{.26, .44, .56}
    \definecolor{gray}{gray}{0.45}
    \definecolor{lightgray}{gray}{.95}
    \definecolor{mediumgray}{gray}{.8}
    \definecolor{inputbackground}{rgb}{.95, .95, .85}
    \definecolor{outputbackground}{rgb}{.95, .95, .95}
    \definecolor{traceback}{rgb}{1, .95, .95}
    % ansi colors
    \definecolor{red}{rgb}{.6,0,0}
    \definecolor{green}{rgb}{0,.65,0}
    \definecolor{brown}{rgb}{0.6,0.6,0}
    \definecolor{blue}{rgb}{0,.145,.698}
    \definecolor{purple}{rgb}{.698,.145,.698}
    \definecolor{cyan}{rgb}{0,.698,.698}
    \definecolor{lightgray}{gray}{0.5}
    
    % bright ansi colors
    \definecolor{darkgray}{gray}{0.25}
    \definecolor{lightred}{rgb}{1.0,0.39,0.28}
    \definecolor{lightgreen}{rgb}{0.48,0.99,0.0}
    \definecolor{lightblue}{rgb}{0.53,0.81,0.92}
    \definecolor{lightpurple}{rgb}{0.87,0.63,0.87}
    \definecolor{lightcyan}{rgb}{0.5,1.0,0.83}
    
    % commands and environments needed by pandoc snippets
    % extracted from the output of `pandoc -s`
    \DefineVerbatimEnvironment{Highlighting}{Verbatim}{commandchars=\\\{\}}
    % Add ',fontsize=\small' for more characters per line
    \newenvironment{Shaded}{}{}
    \newcommand{\KeywordTok}[1]{\textcolor[rgb]{0.00,0.44,0.13}{\textbf{{#1}}}}
    \newcommand{\DataTypeTok}[1]{\textcolor[rgb]{0.56,0.13,0.00}{{#1}}}
    \newcommand{\DecValTok}[1]{\textcolor[rgb]{0.25,0.63,0.44}{{#1}}}
    \newcommand{\BaseNTok}[1]{\textcolor[rgb]{0.25,0.63,0.44}{{#1}}}
    \newcommand{\FloatTok}[1]{\textcolor[rgb]{0.25,0.63,0.44}{{#1}}}
    \newcommand{\CharTok}[1]{\textcolor[rgb]{0.25,0.44,0.63}{{#1}}}
    \newcommand{\StringTok}[1]{\textcolor[rgb]{0.25,0.44,0.63}{{#1}}}
    \newcommand{\CommentTok}[1]{\textcolor[rgb]{0.38,0.63,0.69}{\textit{{#1}}}}
    \newcommand{\OtherTok}[1]{\textcolor[rgb]{0.00,0.44,0.13}{{#1}}}
    \newcommand{\AlertTok}[1]{\textcolor[rgb]{1.00,0.00,0.00}{\textbf{{#1}}}}
    \newcommand{\FunctionTok}[1]{\textcolor[rgb]{0.02,0.16,0.49}{{#1}}}
    \newcommand{\RegionMarkerTok}[1]{{#1}}
    \newcommand{\ErrorTok}[1]{\textcolor[rgb]{1.00,0.00,0.00}{\textbf{{#1}}}}
    \newcommand{\NormalTok}[1]{{#1}}
    
    % Define a nice break command that doesn't care if a line doesn't already
    % exist.
    \def\br{\hspace*{\fill} \\* }
    % Math Jax compatability definitions
    \def\gt{>}
    \def\lt{<}
    % Document parameters
    \title{pres}
    
    
    

    % Pygments definitions
    
\makeatletter
\def\PY@reset{\let\PY@it=\relax \let\PY@bf=\relax%
    \let\PY@ul=\relax \let\PY@tc=\relax%
    \let\PY@bc=\relax \let\PY@ff=\relax}
\def\PY@tok#1{\csname PY@tok@#1\endcsname}
\def\PY@toks#1+{\ifx\relax#1\empty\else%
    \PY@tok{#1}\expandafter\PY@toks\fi}
\def\PY@do#1{\PY@bc{\PY@tc{\PY@ul{%
    \PY@it{\PY@bf{\PY@ff{#1}}}}}}}
\def\PY#1#2{\PY@reset\PY@toks#1+\relax+\PY@do{#2}}

\expandafter\def\csname PY@tok@gd\endcsname{\def\PY@tc##1{\textcolor[rgb]{0.63,0.00,0.00}{##1}}}
\expandafter\def\csname PY@tok@gu\endcsname{\let\PY@bf=\textbf\def\PY@tc##1{\textcolor[rgb]{0.50,0.00,0.50}{##1}}}
\expandafter\def\csname PY@tok@gt\endcsname{\def\PY@tc##1{\textcolor[rgb]{0.00,0.27,0.87}{##1}}}
\expandafter\def\csname PY@tok@gs\endcsname{\let\PY@bf=\textbf}
\expandafter\def\csname PY@tok@gr\endcsname{\def\PY@tc##1{\textcolor[rgb]{1.00,0.00,0.00}{##1}}}
\expandafter\def\csname PY@tok@cm\endcsname{\let\PY@it=\textit\def\PY@tc##1{\textcolor[rgb]{0.25,0.50,0.50}{##1}}}
\expandafter\def\csname PY@tok@vg\endcsname{\def\PY@tc##1{\textcolor[rgb]{0.10,0.09,0.49}{##1}}}
\expandafter\def\csname PY@tok@m\endcsname{\def\PY@tc##1{\textcolor[rgb]{0.40,0.40,0.40}{##1}}}
\expandafter\def\csname PY@tok@mh\endcsname{\def\PY@tc##1{\textcolor[rgb]{0.40,0.40,0.40}{##1}}}
\expandafter\def\csname PY@tok@go\endcsname{\def\PY@tc##1{\textcolor[rgb]{0.53,0.53,0.53}{##1}}}
\expandafter\def\csname PY@tok@ge\endcsname{\let\PY@it=\textit}
\expandafter\def\csname PY@tok@vc\endcsname{\def\PY@tc##1{\textcolor[rgb]{0.10,0.09,0.49}{##1}}}
\expandafter\def\csname PY@tok@il\endcsname{\def\PY@tc##1{\textcolor[rgb]{0.40,0.40,0.40}{##1}}}
\expandafter\def\csname PY@tok@cs\endcsname{\let\PY@it=\textit\def\PY@tc##1{\textcolor[rgb]{0.25,0.50,0.50}{##1}}}
\expandafter\def\csname PY@tok@cp\endcsname{\def\PY@tc##1{\textcolor[rgb]{0.74,0.48,0.00}{##1}}}
\expandafter\def\csname PY@tok@gi\endcsname{\def\PY@tc##1{\textcolor[rgb]{0.00,0.63,0.00}{##1}}}
\expandafter\def\csname PY@tok@gh\endcsname{\let\PY@bf=\textbf\def\PY@tc##1{\textcolor[rgb]{0.00,0.00,0.50}{##1}}}
\expandafter\def\csname PY@tok@ni\endcsname{\let\PY@bf=\textbf\def\PY@tc##1{\textcolor[rgb]{0.60,0.60,0.60}{##1}}}
\expandafter\def\csname PY@tok@nl\endcsname{\def\PY@tc##1{\textcolor[rgb]{0.63,0.63,0.00}{##1}}}
\expandafter\def\csname PY@tok@nn\endcsname{\let\PY@bf=\textbf\def\PY@tc##1{\textcolor[rgb]{0.00,0.00,1.00}{##1}}}
\expandafter\def\csname PY@tok@no\endcsname{\def\PY@tc##1{\textcolor[rgb]{0.53,0.00,0.00}{##1}}}
\expandafter\def\csname PY@tok@na\endcsname{\def\PY@tc##1{\textcolor[rgb]{0.49,0.56,0.16}{##1}}}
\expandafter\def\csname PY@tok@nb\endcsname{\def\PY@tc##1{\textcolor[rgb]{0.00,0.50,0.00}{##1}}}
\expandafter\def\csname PY@tok@nc\endcsname{\let\PY@bf=\textbf\def\PY@tc##1{\textcolor[rgb]{0.00,0.00,1.00}{##1}}}
\expandafter\def\csname PY@tok@nd\endcsname{\def\PY@tc##1{\textcolor[rgb]{0.67,0.13,1.00}{##1}}}
\expandafter\def\csname PY@tok@ne\endcsname{\let\PY@bf=\textbf\def\PY@tc##1{\textcolor[rgb]{0.82,0.25,0.23}{##1}}}
\expandafter\def\csname PY@tok@nf\endcsname{\def\PY@tc##1{\textcolor[rgb]{0.00,0.00,1.00}{##1}}}
\expandafter\def\csname PY@tok@si\endcsname{\let\PY@bf=\textbf\def\PY@tc##1{\textcolor[rgb]{0.73,0.40,0.53}{##1}}}
\expandafter\def\csname PY@tok@s2\endcsname{\def\PY@tc##1{\textcolor[rgb]{0.73,0.13,0.13}{##1}}}
\expandafter\def\csname PY@tok@vi\endcsname{\def\PY@tc##1{\textcolor[rgb]{0.10,0.09,0.49}{##1}}}
\expandafter\def\csname PY@tok@nt\endcsname{\let\PY@bf=\textbf\def\PY@tc##1{\textcolor[rgb]{0.00,0.50,0.00}{##1}}}
\expandafter\def\csname PY@tok@nv\endcsname{\def\PY@tc##1{\textcolor[rgb]{0.10,0.09,0.49}{##1}}}
\expandafter\def\csname PY@tok@s1\endcsname{\def\PY@tc##1{\textcolor[rgb]{0.73,0.13,0.13}{##1}}}
\expandafter\def\csname PY@tok@sh\endcsname{\def\PY@tc##1{\textcolor[rgb]{0.73,0.13,0.13}{##1}}}
\expandafter\def\csname PY@tok@sc\endcsname{\def\PY@tc##1{\textcolor[rgb]{0.73,0.13,0.13}{##1}}}
\expandafter\def\csname PY@tok@sx\endcsname{\def\PY@tc##1{\textcolor[rgb]{0.00,0.50,0.00}{##1}}}
\expandafter\def\csname PY@tok@bp\endcsname{\def\PY@tc##1{\textcolor[rgb]{0.00,0.50,0.00}{##1}}}
\expandafter\def\csname PY@tok@c1\endcsname{\let\PY@it=\textit\def\PY@tc##1{\textcolor[rgb]{0.25,0.50,0.50}{##1}}}
\expandafter\def\csname PY@tok@kc\endcsname{\let\PY@bf=\textbf\def\PY@tc##1{\textcolor[rgb]{0.00,0.50,0.00}{##1}}}
\expandafter\def\csname PY@tok@c\endcsname{\let\PY@it=\textit\def\PY@tc##1{\textcolor[rgb]{0.25,0.50,0.50}{##1}}}
\expandafter\def\csname PY@tok@mf\endcsname{\def\PY@tc##1{\textcolor[rgb]{0.40,0.40,0.40}{##1}}}
\expandafter\def\csname PY@tok@err\endcsname{\def\PY@bc##1{\setlength{\fboxsep}{0pt}\fcolorbox[rgb]{1.00,0.00,0.00}{1,1,1}{\strut ##1}}}
\expandafter\def\csname PY@tok@kd\endcsname{\let\PY@bf=\textbf\def\PY@tc##1{\textcolor[rgb]{0.00,0.50,0.00}{##1}}}
\expandafter\def\csname PY@tok@ss\endcsname{\def\PY@tc##1{\textcolor[rgb]{0.10,0.09,0.49}{##1}}}
\expandafter\def\csname PY@tok@sr\endcsname{\def\PY@tc##1{\textcolor[rgb]{0.73,0.40,0.53}{##1}}}
\expandafter\def\csname PY@tok@mo\endcsname{\def\PY@tc##1{\textcolor[rgb]{0.40,0.40,0.40}{##1}}}
\expandafter\def\csname PY@tok@kn\endcsname{\let\PY@bf=\textbf\def\PY@tc##1{\textcolor[rgb]{0.00,0.50,0.00}{##1}}}
\expandafter\def\csname PY@tok@mi\endcsname{\def\PY@tc##1{\textcolor[rgb]{0.40,0.40,0.40}{##1}}}
\expandafter\def\csname PY@tok@gp\endcsname{\let\PY@bf=\textbf\def\PY@tc##1{\textcolor[rgb]{0.00,0.00,0.50}{##1}}}
\expandafter\def\csname PY@tok@o\endcsname{\def\PY@tc##1{\textcolor[rgb]{0.40,0.40,0.40}{##1}}}
\expandafter\def\csname PY@tok@kr\endcsname{\let\PY@bf=\textbf\def\PY@tc##1{\textcolor[rgb]{0.00,0.50,0.00}{##1}}}
\expandafter\def\csname PY@tok@s\endcsname{\def\PY@tc##1{\textcolor[rgb]{0.73,0.13,0.13}{##1}}}
\expandafter\def\csname PY@tok@kp\endcsname{\def\PY@tc##1{\textcolor[rgb]{0.00,0.50,0.00}{##1}}}
\expandafter\def\csname PY@tok@w\endcsname{\def\PY@tc##1{\textcolor[rgb]{0.73,0.73,0.73}{##1}}}
\expandafter\def\csname PY@tok@kt\endcsname{\def\PY@tc##1{\textcolor[rgb]{0.69,0.00,0.25}{##1}}}
\expandafter\def\csname PY@tok@ow\endcsname{\let\PY@bf=\textbf\def\PY@tc##1{\textcolor[rgb]{0.67,0.13,1.00}{##1}}}
\expandafter\def\csname PY@tok@sb\endcsname{\def\PY@tc##1{\textcolor[rgb]{0.73,0.13,0.13}{##1}}}
\expandafter\def\csname PY@tok@k\endcsname{\let\PY@bf=\textbf\def\PY@tc##1{\textcolor[rgb]{0.00,0.50,0.00}{##1}}}
\expandafter\def\csname PY@tok@se\endcsname{\let\PY@bf=\textbf\def\PY@tc##1{\textcolor[rgb]{0.73,0.40,0.13}{##1}}}
\expandafter\def\csname PY@tok@sd\endcsname{\let\PY@it=\textit\def\PY@tc##1{\textcolor[rgb]{0.73,0.13,0.13}{##1}}}

\def\PYZbs{\char`\\}
\def\PYZus{\char`\_}
\def\PYZob{\char`\{}
\def\PYZcb{\char`\}}
\def\PYZca{\char`\^}
\def\PYZam{\char`\&}
\def\PYZlt{\char`\<}
\def\PYZgt{\char`\>}
\def\PYZsh{\char`\#}
\def\PYZpc{\char`\%}
\def\PYZdl{\char`\$}
\def\PYZhy{\char`\-}
\def\PYZsq{\char`\'}
\def\PYZdq{\char`\"}
\def\PYZti{\char`\~}
% for compatibility with earlier versions
\def\PYZat{@}
\def\PYZlb{[}
\def\PYZrb{]}
\makeatother


    % Exact colors from NB
    \definecolor{incolor}{rgb}{0.0, 0.0, 0.5}
    \definecolor{outcolor}{rgb}{0.545, 0.0, 0.0}



    
    % Prevent overflowing lines due to hard-to-break entities
    \sloppy 
    % Setup hyperref package
    \hypersetup{
      breaklinks=true,  % so long urls are correctly broken across lines
      colorlinks=true,
      urlcolor=blue,
      linkcolor=darkorange,
      citecolor=darkgreen,
      }
    % Slightly bigger margins than the latex defaults
    
    \geometry{verbose,tmargin=1in,bmargin=1in,lmargin=1in,rmargin=1in}
    
    

    \begin{document}
    
    
    \maketitle
    
    

    
    \begin{Verbatim}[commandchars=\\\{\}]
{\color{incolor}In [{\color{incolor}1}]:} \PY{o}{\PYZpc{}}\PY{k}{run} \PY{n}{talktools}
\end{Verbatim}

    
    \begin{verbatim}
<IPython.core.display.HTML at 0x7fd2c4a55510>
    \end{verbatim}

    

    \section{Plotly collaborative, interactive ~ ~ ~ and online plotting with $\TeX$}



    \subsection{What is Plotly?}


    Plotly is an \textbf{online} analytics and data visualization tool.

    \begin{Verbatim}[commandchars=\\\{\}]
{\color{incolor}In [{\color{incolor}2}]:} \PY{k+kn}{from} \PY{n+nn}{IPython.display} \PY{k+kn}{import} \PY{n}{IFrame}
        
        \PY{n}{IFrame}\PY{p}{(}\PY{l+s}{\PYZdq{}}\PY{l+s}{https://plot.ly/\PYZti{}Dreamshot/407}\PY{l+s}{\PYZdq{}}\PY{p}{,} \PY{n}{width}\PY{o}{=}\PY{l+s}{\PYZdq{}}\PY{l+s}{900px}\PY{l+s}{\PYZdq{}}\PY{p}{,} \PY{n}{height}\PY{o}{=}\PY{l+s}{\PYZsq{}}\PY{l+s}{500px}\PY{l+s}{\PYZsq{}}\PY{p}{)}
\end{Verbatim}

            \begin{Verbatim}[commandchars=\\\{\}]
{\color{outcolor}Out[{\color{outcolor}2}]:} <IPython.lib.display.IFrame at 0x7fd2c52ad7d0>
\end{Verbatim}
        

    \subsection{Why is Plotly at TUG 2014?}


    \begin{itemize}
\itemsep1pt\parskip0pt\parsep0pt
\item
  You can use $\TeX$ symbols to annotate Plotly graphs, (using the
  \href{http://www.mathjax.org/}{MathJax} display engine)
\end{itemize}

    \begin{Verbatim}[commandchars=\\\{\}]
{\color{incolor}In [{\color{incolor}3}]:} \PY{c}{\PYZsh{} from IPython.display import IFrame}
        
        \PY{n}{IFrame}\PY{p}{(}\PY{l+s}{\PYZdq{}}\PY{l+s}{https://plot.ly/\PYZti{}PlotBot/3}\PY{l+s}{\PYZdq{}}\PY{p}{,} \PY{n}{width}\PY{o}{=}\PY{l+s}{\PYZdq{}}\PY{l+s}{900px}\PY{l+s}{\PYZdq{}}\PY{p}{,} \PY{n}{height}\PY{o}{=}\PY{l+s}{\PYZsq{}}\PY{l+s}{500px}\PY{l+s}{\PYZsq{}}\PY{p}{)}
\end{Verbatim}

            \begin{Verbatim}[commandchars=\\\{\}]
{\color{outcolor}Out[{\color{outcolor}3}]:} <IPython.lib.display.IFrame at 0x7fd2c52ad710>
\end{Verbatim}
        

    \subsection{But really \ldots{} Why is Plotly at TUG 2014?}



    \paragraph{\ldots{} Philosophy}


    $\TeX$ provides a system that gives exactly the same results on all
computers

\begin{itemize}
\itemsep1pt\parskip0pt\parsep0pt
\item
  $\TeX$ is \textbf{cross-operating system}
\end{itemize}

    $\TeX$ allows anybody to produce high-quality documents efficiently

\begin{itemize}
\itemsep1pt\parskip0pt\parsep0pt
\item
  $\TeX$ is a \textbf{free} and \textbf{cross-text editor} platform
\end{itemize}

    Plotly applies the same core principles to graphs

\begin{itemize}
\itemsep1pt\parskip0pt\parsep0pt
\item
  Plotly is \textbf{free}, \textbf{cross-operating system} and
  \textbf{cross-scientific computing language}
\end{itemize}


    \subsubsection{Collaboration in data-intensive fields sometimes feels like this:}


    

    Plotly solves this collaboration (i.e.~reproducibility) problem:

\begin{itemize}
\itemsep1pt\parskip0pt\parsep0pt
\item
  Plotly graphs are \textbf{closely connected to their underlying data}
  (more later)
\item
  Plotly graphs are \textbf{stored in the cloud}
\item
  Plotly provides a \textbf{common graphing platform} for Python,
  MATLAB, R, Node.js, Julia and Excel users.
\end{itemize}


    \subsubsection{A Plotly graph made in Python}


    \begin{Verbatim}[commandchars=\\\{\}]
{\color{incolor}In [{\color{incolor}4}]:} \PY{c}{\PYZsh{} from IPython.display import IFrame}
        
        \PY{n}{IFrame}\PY{p}{(}\PY{l+s}{\PYZdq{}}\PY{l+s}{https://plot.ly/python/histograms/\PYZsh{}Overlaid\PYZhy{}Histgram}\PY{l+s}{\PYZdq{}}\PY{p}{,} \PY{n}{width}\PY{o}{=}\PY{l+s}{\PYZdq{}}\PY{l+s}{1100px}\PY{l+s}{\PYZdq{}}\PY{p}{,} \PY{n}{height}\PY{o}{=}\PY{l+s}{\PYZsq{}}\PY{l+s}{520px}\PY{l+s}{\PYZsq{}}\PY{p}{)}
\end{Verbatim}

            \begin{Verbatim}[commandchars=\\\{\}]
{\color{outcolor}Out[{\color{outcolor}4}]:} <IPython.lib.display.IFrame at 0x7fd2c52b6fd0>
\end{Verbatim}
        

    \subsubsection{The same Plotly graph, now made in MATLAB}


    \begin{Verbatim}[commandchars=\\\{\}]
{\color{incolor}In [{\color{incolor}5}]:} \PY{c}{\PYZsh{} from IPython.display import IFrame}
        
        \PY{n}{IFrame}\PY{p}{(}\PY{l+s}{\PYZdq{}}\PY{l+s}{https://plot.ly/matlab/histograms/\PYZsh{}Overlaid\PYZhy{}Histgram}\PY{l+s}{\PYZdq{}}\PY{p}{,} \PY{n}{width}\PY{o}{=}\PY{l+s}{\PYZdq{}}\PY{l+s}{1100px}\PY{l+s}{\PYZdq{}}\PY{p}{,} \PY{n}{height}\PY{o}{=}\PY{l+s}{\PYZsq{}}\PY{l+s}{520px}\PY{l+s}{\PYZsq{}}\PY{p}{)}
\end{Verbatim}

            \begin{Verbatim}[commandchars=\\\{\}]
{\color{outcolor}Out[{\color{outcolor}5}]:} <IPython.lib.display.IFrame at 0x7fd2c4a55210>
\end{Verbatim}
        

    \subsubsection{\ldots{} and again the same Plotly graph, now made in R}


    \begin{Verbatim}[commandchars=\\\{\}]
{\color{incolor}In [{\color{incolor}6}]:} \PY{c}{\PYZsh{} from IPython.display import IFrame}
        
        \PY{n}{IFrame}\PY{p}{(}\PY{l+s}{\PYZdq{}}\PY{l+s}{https://plot.ly/r/histograms/\PYZsh{}Overlaid\PYZhy{}Histgram}\PY{l+s}{\PYZdq{}}\PY{p}{,} \PY{n}{width}\PY{o}{=}\PY{l+s}{\PYZdq{}}\PY{l+s}{1100px}\PY{l+s}{\PYZdq{}}\PY{p}{,} \PY{n}{height}\PY{o}{=}\PY{l+s}{\PYZsq{}}\PY{l+s}{520px}\PY{l+s}{\PYZsq{}}\PY{p}{)}
\end{Verbatim}

            \begin{Verbatim}[commandchars=\\\{\}]
{\color{outcolor}Out[{\color{outcolor}6}]:} <IPython.lib.display.IFrame at 0x7fd2c52adc50>
\end{Verbatim}
        

    \subsubsection{What if I already have plot-generating code written?}


    That's OK.

Our libraries come with \textbf{figure converters} allowing

\begin{itemize}
\itemsep1pt\parskip0pt\parsep0pt
\item
  \href{https://plot.ly/matlab/}{MATLAB},
\item
  \href{https://plot.ly/matplotlib/}{matplotlib} and
\item
  \href{https://plot.ly/ggplot2/}{ggplot2}
\end{itemize}

figures to be converted to Plotly figures with one line of code!

    \begin{Verbatim}[commandchars=\\\{\}]
{\color{incolor}In [{\color{incolor}7}]:} \PY{c}{\PYZsh{} from IPython.display import IFrame}
        
        \PY{n}{IFrame}\PY{p}{(}\PY{l+s}{\PYZdq{}}\PY{l+s}{https://plot.ly/matplotlib/}\PY{l+s}{\PYZdq{}}\PY{p}{,} \PY{n}{width}\PY{o}{=}\PY{l+s}{\PYZdq{}}\PY{l+s}{1100px}\PY{l+s}{\PYZdq{}}\PY{p}{,} \PY{n}{height}\PY{o}{=}\PY{l+s}{\PYZsq{}}\PY{l+s}{520px}\PY{l+s}{\PYZsq{}}\PY{p}{)}
\end{Verbatim}

            \begin{Verbatim}[commandchars=\\\{\}]
{\color{outcolor}Out[{\color{outcolor}7}]:} <IPython.lib.display.IFrame at 0x7fd2c4a55250>
\end{Verbatim}
        

    \subsubsection{How to make Plotly graph using our web app}


    \begin{itemize}
\itemsep1pt\parskip0pt\parsep0pt
\item
  Have some
  \href{http://en.wikipedia.org/wiki/Portland,_Oregon\#Demographics}{data}
  to plot
\end{itemize}

    \begin{itemize}
\itemsep1pt\parskip0pt\parsep0pt
\item
  Go to \href{https://plot.ly}{plot.ly}
\end{itemize}


    \subsubsection{Plotly rhymes with collaboration and reproducibility}


    Plotly allows you to \textbf{retrieve} a figure's underlying JSON
object!

For example, in Python:

    \begin{Verbatim}[commandchars=\\\{\}]
{\color{incolor}In [{\color{incolor}8}]:} \PY{k+kn}{import} \PY{n+nn}{plotly.plotly} \PY{k+kn}{as} \PY{n+nn}{py}
        
        \PY{n}{fig} \PY{o}{=} \PY{n}{py}\PY{o}{.}\PY{n}{get\PYZus{}figure}\PY{p}{(}\PY{l+s}{\PYZdq{}}\PY{l+s}{https://plot.ly/\PYZti{}etpinard/448}\PY{l+s}{\PYZdq{}}\PY{p}{)}
        
        \PY{n}{fig}
\end{Verbatim}

            \begin{Verbatim}[commandchars=\\\{\}]
{\color{outcolor}Out[{\color{outcolor}8}]:} \{'data': [\{'name': u'Pop.',
           'type': u'scatter',
           'x': [u'1860',
            u'1870',
            u'1880',
            u'1890',
            u'1900',
            u'1910',
            u'1920',
            u'1930',
            u'1940',
            u'1950',
            u'1960',
            u'1970',
            u'1980',
            u'1990',
            u'2000',
            u'2010',
            u'2013'],
           'y': [u'2,874',
            u'8,293',
            u'17,577',
            u'46,385',
            u'90,426',
            u'207,214',
            u'258,288',
            u'301,815',
            u'305,394',
            u'373,628',
            u'372,676',
            u'382,619',
            u'366,383',
            u'437,319',
            u'529,121',
            u'583,776',
            u'609,456']\},
          \{'line': \{'color': u'rgb(255, 127, 14)', 'dash': u'solid', 'width': 6\},
           'name': u'Pop. - fit',
           'opacity': 0.5,
           'type': u'scatter',
           'x': [1860,
            1863.1224489795918,
            1866.2448979591836,
            1869.3673469387754,
            1872.4897959183672,
            1875.6122448979593,
            1878.734693877551,
            1881.857142857143,
            1884.9795918367347,
            1888.1020408163265,
            1891.2244897959183,
            1894.3469387755101,
            1897.469387755102,
            1900.591836734694,
            1903.7142857142858,
            1906.8367346938776,
            1909.9591836734694,
            1913.0816326530612,
            1916.204081632653,
            1919.3265306122448,
            1922.4489795918366,
            1925.5714285714287,
            1928.6938775510205,
            1931.8163265306123,
            1934.938775510204,
            1938.061224489796,
            1941.1836734693877,
            1944.3061224489795,
            1947.4285714285713,
            1950.5510204081634,
            1953.6734693877552,
            1956.795918367347,
            1959.9183673469388,
            1963.0408163265306,
            1966.1632653061224,
            1969.2857142857142,
            1972.408163265306,
            1975.530612244898,
            1978.6530612244899,
            1981.7755102040817,
            1984.8979591836735,
            1988.0204081632653,
            1991.142857142857,
            1994.265306122449,
            1997.3877551020407,
            2000.5102040816328,
            2003.6326530612246,
            2006.7551020408164,
            2009.8775510204082,
            2013],
           'xaxis': u'x',
           'y': [-24905.93916280754,
            -12636.203280905262,
            -366.4673990039155,
            11903.268482898362,
            24173.00436480064,
            36442.74024670292,
            48712.476128605194,
            60982.21201050747,
            73251.94789240882,
            85521.6837743111,
            97791.41965621337,
            110061.15553811472,
            122330.891420017,
            134600.6273019202,
            146870.36318382155,
            159140.09906572383,
            171409.8349476261,
            183679.57082952745,
            195949.30671142973,
            208219.042593332,
            220488.77847523335,
            232758.51435713656,
            245028.25023903884,
            257297.9861209402,
            269567.72200284246,
            281837.45788474474,
            294107.1937666461,
            306376.92964854836,
            318646.66553045064,
            330916.4014123529,
            343186.1372942552,
            355455.8731761575,
            367725.6090580588,
            379995.3449399611,
            392265.0808218634,
            404534.81670376565,
            416804.552585667,
            429074.2884675702,
            441344.0243494725,
            453613.76023137383,
            465883.4961132761,
            478153.2319951784,
            490422.96787707973,
            502692.703758982,
            514962.4396408843,
            527232.1755227866,
            539501.9114046888,
            551771.6472865911,
            564041.3831684925,
            576311.1190503947],
           'yaxis': u'y'\}],
         'layout': \{'annotations': [\{'align': u'left',
            'arrowcolor': u'\#636363',
            'arrowhead': 2,
            'arrowsize': 1,
            'arrowwidth': 2,
            'ax': -265,
            'ay': -112,
            'bgcolor': u'rgba(0,0,0,0)',
            'bordercolor': u'',
            'borderpad': 1,
            'borderwidth': 1,
            'font': \{'size': 28\},
            'opacity': 0.8,
            'showarrow': True,
            'text': u'\$R\^{}2 = 0.960 \textbackslash{}\textbackslash{}\textbackslash{}\textbackslash{} y   = -7.33\textbackslash{}\textbackslash{}times10\^{}6 + 3.93\textbackslash{}\textbackslash{}times10\^{}3 x\$',
            'textangle': 0,
            'x': 1933.706238213492,
            'xanchor': u'auto',
            'xref': u'x',
            'y': 262636.0070362699,
            'yanchor': u'auto',
            'yref': u'y'\}],
          'autosize': True,
          'bargap': 0.2,
          'bargroupgap': 0,
          'barmode': u'group',
          'boxmode': u'overlay',
          'dragmode': u'zoom',
          'font': \{'color': u'\#444',
           'family': u'"Open sans", verdana, arial, sans-serif',
           'size': 12\},
          'height': 474,
          'hidesources': False,
          'hovermode': u'x',
          'legend': \{'bgcolor': u'\#fff',
           'bordercolor': u'\#444',
           'borderwidth': 0,
           'font': \{'color': u'', 'family': u'', 'size': 0\},
           'traceorder': u'normal',
           'x': 1.02,
           'xanchor': u'left',
           'y': 1,
           'yanchor': u'top'\},
          'margin': \{'autoexpand': True,
           'b': 80,
           'l': 80,
           'pad': 0,
           'r': 80,
           't': 100\},
          'paper\_bgcolor': u'\#fff',
          'plot\_bgcolor': u'\#fff',
          'separators': u'.,',
          'showlegend': True,
          'title': u"Portland's Pop",
          'titlefont': \{'color': u'', 'family': u'', 'size': 0\},
          'width': 1297,
          'xaxis': \{'anchor': u'y',
           'autorange': True,
           'autotick': True,
           'domain': [0, 1],
           'dtick': 20,
           'exponentformat': u'B',
           'gridcolor': u'\#eee',
           'gridwidth': 1,
           'linecolor': u'\#444',
           'linewidth': 1,
           'mirror': False,
           'nticks': 0,
           'overlaying': False,
           'position': 0,
           'range': [1851.5, 2021.5],
           'rangemode': u'normal',
           'showexponent': u'all',
           'showgrid': True,
           'showline': False,
           'showticklabels': True,
           'tick0': 0,
           'tickangle': u'auto',
           'tickcolor': u'\#444',
           'tickfont': \{'color': u'', 'family': u'', 'size': 0\},
           'ticklen': 5,
           'ticks': u'',
           'tickwidth': 1,
           'title': u'Census',
           'titlefont': \{'color': u'', 'family': u'', 'size': 0\},
           'type': u'linear',
           'zeroline': True,
           'zerolinecolor': u'\#444',
           'zerolinewidth': 1\},
          'yaxis': \{'anchor': u'x',
           'autorange': True,
           'autotick': True,
           'domain': [0, 1],
           'dtick': 100000,
           'exponentformat': u'B',
           'gridcolor': u'\#eee',
           'gridwidth': 1,
           'linecolor': u'\#444',
           'linewidth': 1,
           'mirror': False,
           'nticks': 0,
           'overlaying': False,
           'position': 0,
           'range': [-60148.26911629685, 644698.3299534894],
           'rangemode': u'normal',
           'showexponent': u'all',
           'showgrid': True,
           'showline': False,
           'showticklabels': True,
           'tick0': 0,
           'tickangle': u'auto',
           'tickcolor': u'\#444',
           'tickfont': \{'color': u'', 'family': u'', 'size': 0\},
           'ticklen': 5,
           'ticks': u'',
           'tickwidth': 1,
           'title': u'Pop.',
           'titlefont': \{'color': u'', 'family': u'', 'size': 0\},
           'type': u'-',
           'zeroline': True,
           'zerolinecolor': u'\#444',
           'zerolinewidth': 1\}\}\}
\end{Verbatim}
        

    \subsubsection{Remake this plot, with a few modifications}


    \begin{Verbatim}[commandchars=\\\{\}]
{\color{incolor}In [{\color{incolor}9}]:} \PY{k+kn}{import} \PY{n+nn}{plotly.plotly} \PY{k+kn}{as} \PY{n+nn}{py}
        \PY{n}{fig} \PY{o}{=} \PY{n}{py}\PY{o}{.}\PY{n}{get\PYZus{}figure}\PY{p}{(}\PY{l+s}{\PYZdq{}}\PY{l+s}{https://plot.ly/\PYZti{}etpinard/448}\PY{l+s}{\PYZdq{}}\PY{p}{)}  \PY{c}{\PYZsh{} as in last slide}
        
        \PY{c}{\PYZsh{} Modify the title}
        \PY{n}{fig}\PY{p}{[}\PY{l+s}{\PYZsq{}}\PY{l+s}{layout}\PY{l+s}{\PYZsq{}}\PY{p}{]}\PY{o}{.}\PY{n}{update}\PY{p}{(}\PY{n}{title}\PY{o}{=}\PY{l+s}{\PYZdq{}}\PY{l+s}{The Historical Population of Portland, OR}\PY{l+s}{\PYZdq{}}\PY{p}{)}
        
        \PY{c}{\PYZsh{} Modify the y\PYZhy{}axis label}
        \PY{n}{fig}\PY{p}{[}\PY{l+s}{\PYZsq{}}\PY{l+s}{layout}\PY{l+s}{\PYZsq{}}\PY{p}{]}\PY{p}{[}\PY{l+s}{\PYZsq{}}\PY{l+s}{yaxis}\PY{l+s}{\PYZsq{}}\PY{p}{]}\PY{o}{.}\PY{n}{update}\PY{p}{(}\PY{n}{title}\PY{o}{=}\PY{l+s}{\PYZdq{}}\PY{l+s}{Population}\PY{l+s}{\PYZdq{}}\PY{p}{)}
        
        \PY{c}{\PYZsh{} Plots the data with marker points (not line)}
        \PY{n}{fig}\PY{p}{[}\PY{l+s}{\PYZsq{}}\PY{l+s}{data}\PY{l+s}{\PYZsq{}}\PY{p}{]}\PY{p}{[}\PY{l+m+mi}{0}\PY{p}{]}\PY{o}{.}\PY{n}{update}\PY{p}{(}\PY{n}{mode}\PY{o}{=}\PY{l+s}{\PYZsq{}}\PY{l+s}{markers}\PY{l+s}{\PYZsq{}}\PY{p}{)}
        
        
        \PY{c}{\PYZsh{} Re\PYZhy{}generate plot, get a unique URL}
        \PY{n}{py}\PY{o}{.}\PY{n}{plot}\PY{p}{(}\PY{n}{fig}\PY{p}{,} \PY{n}{filename}\PY{o}{=}\PY{l+s}{\PYZdq{}}\PY{l+s}{tug\PYZhy{}conf\PYZhy{}ex}\PY{l+s}{\PYZdq{}}\PY{p}{)}
\end{Verbatim}

            \begin{Verbatim}[commandchars=\\\{\}]
{\color{outcolor}Out[{\color{outcolor}9}]:} u'https://plot.ly/\textasciitilde{}etpinard/449'
\end{Verbatim}
        
    Go to plot's URL (or use \texttt{py.iplot()} to embed plot in IPython
notebook)


    \subsection{Moreover, Plotly}


    \begin{itemize}
\itemsep1pt\parskip0pt\parsep0pt
\item
  is also a social network (with a twitter-like \emph{feed} of figures
  and \href{https://plot.ly/~etpinard/25}{commenting} on each graphs)
\item
  allows users to make \emph{private} figures (kind of like github, see
  our \href{https://plot.ly/plans}{plans})
\item
  allows users to make \emph{streaming} plots (e.g.~a never-ending
  double pendulum simulation:
  \href{https://plot.ly/~streaming-demos/4}{graph} and
  \href{http://nbviewer.ipython.org/github/plotly/python-user-guide/blob/master/s7_streaming/s7_streaming_p2-double-pendulum.ipynb}{code})
\item
  is developing of Open Source Libraries (e.g
  \href{https://github.com/plotly/python-api}{Python},
  \href{https://github.com/plotly/matlab-api}{MATLAB},
  \href{https://github.com/ropensci/plotly}{R})
\end{itemize}


    \paragraph{Thank you.}



    % Add a bibliography block to the postdoc
    
    
    
    \end{document}
