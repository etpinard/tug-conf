\documentclass[11pt]{article} 
\usepackage[letterpaper,%
						top=0.25in,bottom=0.5in,right=0.75in,left=0.75in,%
						includehead,includefoot]{geometry}
\usepackage{misc}                         % configuration de la page
\usepackage[utf8x]{inputenc}              % encodage
\usepackage{mypazo}                       % police 'pazo'
\usepackage{avant}                        % police pour sans serif
\usepackage{bbding}                       % symboles
\usepackage{mymath}                       % math 
\usepackage{xspace}                       % espace après raccourcis 
\usepackage{graphicx}                     % graphiques + '\scalebox'
\usepackage[usenames,dvipsnames]{xcolor}
\usepackage{hyperref}
	\hypersetup{colorlinks=false,linkcolor=red,
		linkbordercolor=red,pdfborderstyle={/S/U/W 1}}
\usepackage[round]{natbib}
% -------------------------------------------------------------------------------

% interligne, numérotage et format pour boîtes d'encadrement 
\linespread{1.2}
\pagestyle{empty}
\setlength{\fboxsep}{8pt}
% -------------------------------------------------------------------------------

% raccourcis (texte et symboles)
\def\name{Étienne Tétreault-Pinard}
\def\address{1850 boul. St-Joseph Est app.5 Montréal QC}
\def\phone{514-267-2118}
\newcommand{\mcgill}{McGill University\xspace}
\newcommand{\uw}{University of Washington\xspace}
\def\email{\href{mailto:etienne.t.pinard@gmail.co}{etienne.t.pinard@gmail.com}}
\def\plot{\href{https://plot.ly/~etpinard/25/}
               {here}\xspace}
\def\nbvi{\href{http://nbviewer.ipython.org/gist/etpinard/9278679}
               {IPython notebook}\xspace}
% -------------------------------------------------------------------------------

% -------------------------------------------------------------------------------
% -------------------------------------------------------------------------------
% -------------------------------------------------------------------------------

\begin{document}

\begin{center}
{\LARGE\textsf{TUG 2014}}
\bump\large{\textsf{\today}}
\end{center}
\bigskip \hrule \bigskip\medskip

% title
{\large \textbf{Title:~} 
%
Plotly: collaborative, interactive, and online plotting with \TeX \medskip
% -------------------------------------------------------------------------------

% abstract
\textbf{Abstract} 

\TeX was designed with the goal of allowing anyone to produce high-quality
documents with minimal effort, and to provide a system
compatible on all computers, now and in the future \citep{gaudeul06}. 

Plotly applies the same core principles to graphics. Plotly allows users
collaboratively make and share interactive graphics online using Python, MATLAB,
R, Excel data and \TeX (MathJax) for free. 
%
Additionally, Plotly allows users to easily embed and export graphics for
publication, while backing up your graphics, data and revisions in the cloud.
This tutorial outlines Plotly’s features and demonstrates how using Plotly
creates unique workflows with emphasis on collaboration and reproducibility.

For more \,\href{http://bit.ly/1vdF6Kp}{\texttt{bit.ly/1vdF6Kp}}.

\medskip

% bibliography
\renewcommand\bibsection{\large\textbf{\refname}\mmedskip}
\bibliographystyle{ametsoc}       % last name, first name initial, year ---
\bibliography{tug}


\end{document}


